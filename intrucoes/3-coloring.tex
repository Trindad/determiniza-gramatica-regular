\documentclass[10pt,a4paper,titlepage]{coursepaper}

\usepackage[english]{babel}
\usepackage[utf8x]{inputenc}
\usepackage[T1]{fontenc}
\usepackage{kpfonts} % load a font with all the characters

\usepackage{hyperref}
\usepackage{amsthm}
\usepackage{amssymb}
\usepackage{bm}
\usepackage{graphicx,url}


\author{Silvana Trindade, Maurício André Cinelli}
\college{Universidade Federal da Fronteira Sul}
\coursename{Ciência da Computação}
\coursesection{Linguagens Formais e Autômatos}
\coursenumber{001}
\instructor{Braulio Adriano de Melo}
\studentnumber{1121101051,1121101005}
\coursenumber{}

\title{INSTRUÇÕES DO ALGORITMO}

\begin{document}
\maketitle

\section *{Instruções}

\subsection *{Compilação}

Para compilar o programa, deverá ser executado o seguinte comando em linha de comando:

\begin{center}
gcc main.c -o determiniza -Wall
\end{center}

\subsection *{Execução}

O programa espera um arquivo de teste como sua entrada para executar.
O formato do mesmo se dá a seguir:\\
$a,b,c$\\
$S:aA|bB|cS|c|b$\\
$A:aS|a|bC|cA$\\
$B:aA|cB|cS|c$\\
$C:aS|a|cA|cC$\\
$\$\$$


A primeira linha denota os símbolos do alfabeto, separados por vírgula.
Após isso é inserido a gramática no formato $[ESTADO]:PRODUCAO1|PRODUCAO2$.

O primeiro estado digitado é o estado de entrada da gramática.
Para determinizarmos, fazemos uso de um estado reservado, denotado por $X$, onde este não é possível ser usado como entrada direta da gramática.

Após os estados serem digitados, é necessário digitar $\$\$$ para demarcar o fim
da leitura da gramática.

Após isto, o programa irá determinizar a gramática, se necessário.

Supondo que o arquivo de teste criado tenha o nome $teste2$, então para executar o algoritmo, deve-se digitar o seguinte comando em um terminal:

\begin{center}
./determiniza < teste2
\end{center}

Caso seja necessário determinizar a gramática, a saída do algoritmo será uma tabela de transições da gramática determinizada.

\end{document}