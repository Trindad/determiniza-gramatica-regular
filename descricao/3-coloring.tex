\documentclass[10pt,a4paper,titlepage]{hitec}

\usepackage[brazil]{babel}
\usepackage[utf8x]{inputenc}
\usepackage[T1]{fontenc}
\usepackage{kpfonts} % load a font with all the characters

\usepackage{hyperref}
\usepackage{amsthm}
\usepackage{amssymb}
\usepackage{bm}
\usepackage{graphicx,url}


\author{Silvana Trindade e Maurício André Cinelli}
\title{DETERMINIZAÇÃO DE GRAMÁTICAS REGULARES }

\begin{document}
\maketitle

\section*{Linguagens Regulares}
As linguagens regulares são constituídas de um conjunto de linguagens decidíveis simples e com propriedades bem definidas e compreendidas. 
Essas linguagens podem ser reconhecidas por autômatos finitos e são facilmente descritas por expressões simples, chamadas expressões regulares (ER).

O estudo das linguagens regulares pode ser abordado através de três diferentes formalismos:

\begin{itemize}
\item \textit{operacional ou reconhecedor}: Autômato Finito, que pode ser determinístico,
não determinístico ou com movimento vazio (com $\epsilon$-transição).
\item \textit{axiomático ou gerador}: Gramática Regular.
\item \textit{denotacional}: Expressão Regular (também pode ser considerado gerador).
\end{itemize}

\section*{Autômatos Finitos}

Um autômato finito, pode ser vista como uma máquina composta basicamente por três
partes:

\begin{itemize}
\item \textit{Fita}: Dispositivo de entrada que contém a informação a ser processada. A fita é finita à esquerda e à direita. 
É dividida em células onde cada uma armazena um símbolo. Os símbolos pertencem a um alfabeto de entrada.
Não é possível gravar sobre a fita.
Não existe memória auxiliar. Inicialmente a palavra a ser processada, isto é, a informação de entrada ocupa toda a fita.

\item \textit{Unidade de Controle}: Reflete o estado corrente da máquina.
Possui uma unidade de leitura (cabeça de leitura), que acessa uma unidade da fita de cada
vez. 
Pode assumir um número finito e pré-definido de estados.
Após cada leitura a cabeça move-se uma célula para a direita.

\item \textit{Programa ou Função de Transição}: Função que comanda as leituras e define o
estado da máquina.
Dependendo do estado corrente e do símbolo lido determina o novo estado do autômato. 
Usa-se o conceito de estado para armazenar as informações necessárias à determinação do próximo estado, uma vez que não há memória auxiliar.

\end{itemize}

\section*{Autômato Finito Determinístico}

Um autômato finito determinístico (AFD), ou simplesmente autômato finito, M é uma
quíntupla:
\bigskip

\begin{center}
$M = (\Sigma,Q,\delta,q_0,F)$
\end{center}

onde:

\begin{description}
\item[$\Sigma$] - Alfabeto de símbolos de entrada 
\item[$Q$] - Conjunto finito de estados possíveis do autômato.
\item[$\delta$] - Função programa ou função de transição $\delta$: $Q \times \Sigma \rightarrow Q$.
\item[$q_0$] - Estado inicial tal que $q0 \in Q$.
\item[$F$] - Conjunto de estados finais, tais que $F \subseteq Q$.
\end{description}

\section*{Autômato Finito Não-Determinístico (AFND)}

\begin{itemize}
\item Não-determinismo é uma importante generalização dos AF's, essencial para a teoria da computação e para a teoria das linguagens formais.
\item Qualquer AFND pode ser simulado por um autômato finito determinístico.
Em AFNDs, a função programa leva de um par (estado, símbolo) a um conjunto de estados possíveis.
\item Em AFNDs, a função programa leva de um par (estado, símbolo) a um conjunto
de estados possíveis.
\item Pode-se entender que o AFND assume simultaneamente todas as alternativas de
estados possíveis \{ $p_0, p_1, ..., p_n$ \} a partir do estado atual ($q \in Q$) e do símbolo
recebido ($a \in \Sigma $ ), como se houvesse uma unidade de controle para processar
cada alternativa independentemente, sem compartilhar recursos com as demais.
\item Assim o processamento de um caminho não influi no estado, símbolo lido e
posição da cabeça dos demais caminhos alternativos.
\end{itemize}

\section*{processo de Determinização}

Por definição, todo AFD é um caso especial de AFND no qual a relação de transição é uma função.
Assim, a classe de linguagens reconhecidas por um AFND inclui as linguagens regulares (aquelas que são reconhecidas por AFD's).
Entretanto, pode-se provar que as linguagens regulares são as únicas linguagens reconhecidas por um AFND. 
Para isto, basta mostrar que para qualquer AFND pode-se construir um AFD que reconhece a mesma linguagem. 
Um método de transformação é dado a seguir.

\subsection*{Algoritmo}

O algoritmo de determinização tem como entrada o alfabeto da linguagem e a gramática a ser determinizada.

O algoritmo verifica se a gramatica é regular e se é deterministica ou indeterministica. Se a mesma for deterministica ou não regular, então o programa para, pois não há razão para continuar  a execução do algoritmo.

Do contrário, o processo de determinização é iniciado, percorrendo todos os estados da AFND, verificando se há mais de um caminho para estados a partir de cada simbolo do alfabeto. Se houver mais de um caminho, significa que há uma indeterminização. Para resolver isso, é necessário criar um novo estado, onde as transiç~oes de ambos os estados anteriores serão inseridas.

Este processo é executado até que todos os estados já visitados pelo algoritmo sejam inseridos na nova gramática determinizada. Sendo assim, é possível que alguns estados passem a não mais existir, e que muitos outros sejam criados durante o processo.

\section*{Estratégias}

Optamos por fazer a leitura da gramática e não o autômato diretamente, pois é mais genérica, mas também porque é mais simples de o usuário digitar a gramática. Além disso, o usuário consegue ver os processos no qual a mesma passa para ser determinizada, que é determinizar usando o autômato finito.	

Antes de determinizarmos, verificamos se a gramática é regular e se é indeterminística, pois nesses casos não é necessário ou mesmo não é possível realizar o processo de determinização com o nosso algoritmo.

\end{document}